%!TEX root = ../template.tex
%%%%%%%%%%%%%%%%%%%%%%%%%%%%%%%%%%%%%%%%%%%%%%%%%%%%%%%%%%%%%%%%%%%
%% chapterIntroduction.tex
%% NOVA thesis document file
%%
%% Chapter with introduction
%%%%%%%%%%%%%%%%%%%%%%%%%%%%%%%%%%%%%%%%%%%%%%%%%%%%%%%%%%%%%%%%%%%

\typeout{NT FILE chapterIntroduction.tex}%

\part{Introductory material}

\chapter{Introduction}
\label{cha:introduction}

\prependtographicspath{{Chapters/Figures/Covers/}}

% epigraph configuration
\epigraphfontsize{\small\itshape}
\setlength\epigraphwidth{12.5cm}
\setlength\epigraphrule{0pt}

\section{Contextualization}
\label{sec:contextualization}

Nowadays, with society's increasing dependence on IT systems, starting on a personal level and building up to an economic level,
 the need for secure systems is more important than ever. This digital transformation of society was intensified by the COVID-19 pandemic,
 which expanded the threat landscape and brought new challenges to the cybersecurity field, which required adapted and innovative solutions.
The number of cyber-attacks continues to rise daily, with increasingly sophisticated attacks making it harder and harder to protect IT systems\cite{cybersecurity-strategy}.
While IT systems are the applications, software, and services that we rely on daily, the IT infrastructure is the backbone of these systems,
 encompassing the hardware, networks, data centers, and security frameworks that support and enable IT systems to function securely and efficiently.
There for the need to secure IT infrastructure is crucial to ensure the confidentiality, integrity, and availability of IT systems and their critical data.
That's where risk assessment tools come in. These tools act as passive security measures and although they are not the principal responsible for preventing attacks,
they act as the first line of defense responsible for identifying, classifying, and mitigating risks if they are considered critical for that infrastructure.
This is one of many measures that can be used and abused to make IT infrastructures more robust and secure.
\par Economically speaking the cost of a cyber-attack can be devastating, in 2020 the global economic cost of cybercrime was 5.5 trillion euros\cite{cybercrime-cost},
 reaching twice the level of 2015. This is a clear indicator that the cost of cybercrime is increasing and that the need for secure IT infrastructures is more important than ever,
 that is why the boosting of cybersecurity is essential. The principal attacks in 2024 on the European Union were, from first to last, Ransomware, Malware, and Social Engineering\cite{ENISA-Threat-landscape-2024}.
The strategy to combat these according to the European Commission is to deploy three principal instruments, firstly resilience, technological sovereignty and global leadership.
Secondly, operational capacity to prevent, deter and respond, and lastly, cooperation to advance a global and open cyberspace\cite{cybersecurity-strategy}.




\subsection{Objectives}
\label{sub:objectives}

This master's dissertation aims to compare and evaluate the most relevant risk assessment tools in the market, focusing on the tools that are most used in the industry
 for the evaluation of IT infrastructures, with special emphasis on compliance with the NIS2 Directive and operational efficiency. A selection will be carried out and also an
 evaluation of widely used solutions with a focus on their applicability in the context of infrastructure security and resilience requirements. The methodology will involve
 the development of evaluation criteria specific for the analysis of different tools, considering the integration capacity, ease of use, adaptability to the needs of different
 infrastructures, and alignment with the regulatory requirements, as well other factors. The results of this study will provide a detailed comparative analysis and also a ranking
 of the most efficient and regulatory-compliant tools, based on practical simulations and risk scenarios. With this research is expected to contribute valuable insights for
 organizations that want to optimize risk assessment and management processes in their IT infrastructures, ensuring compliance with the NIS2 Directive.


