%!TEX root = ../template.tex
%%%%%%%%%%%%%%%%%%%%%%%%%%%%%%%%%%%%%%%%%%%%%%%%%%%%%%%%%%%%%%%%%%%%
%% abstract-en.tex
%% NOVA thesis document file
%%
%% Abstract in English([^%]*)
%%%%%%%%%%%%%%%%%%%%%%%%%%%%%%%%%%%%%%%%%%%%%%%%%%%%%%%%%%%%%%%%%%%%

\typeout{NT FILE abstract-en.tex}%

With the growth of complexity in IT Infrastructures and the increase of cyber threats, increasingly require the use of robust risk management practices
 to ensure safety and resilience of systems in a enterprise context. With the implementation of the NIS2 Directive in European Union, has become
 an obligation to adopt risk assessment mechanisms to ensure regulatory compliance and efficient protection of critical assets. This work aims to explore
 and compare the main risk assessment tools available at this moment, with emphasis on the their operational efficiency and ability to fulfill the requirements imposed by NIS2.
Through a comparative analysis it is intended to identify which solutions best adapt to security needs of nowadays infraestructures,
 such as \textit{Data Centers} e \textit{Enterprise Networks}. This study aims for an in-depth understanding of the best risk assessment practices that balance regulatory compliance and efficiency, 
allowing organizations to adopt more resilient and aligned strategies with cybersecurity requirements regarding infraestructure.

\keywords{
  IT Infrastructures \and
  Cyber threats \and
  Risk management \and
  Safety \and
  Resilience \and
  Enterprise context \and  
  NIS2 Directive \and 
  European Union \and 
  Regulatory compliance \and 
  Risk assessment \and 
  Critical assets \and 
  Operational efficiency \and 
  Comparative analysis \and 
  Security needs \and 
  Data Centers \and 
  Enterprise Networks \and 
  Cybersecurity \and 
  Resilient strategies
}
