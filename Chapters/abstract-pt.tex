%!TEX root = ../template.tex
%%%%%%%%%%%%%%%%%%%%%%%%%%%%%%%%%%%%%%%%%%%%%%%%%%%%%%%%%%%%%%%%%%%%
%% abstract-pt.tex
%% NOVA thesis document file
%%
%% Abstract in Portuguese
%%%%%%%%%%%%%%%%%%%%%%%%%%%%%%%%%%%%%%%%%%%%%%%%%%%%%%%%%%%%%%%%%%%%

\typeout{NT FILE abstract-pt.tex}%

O grande crescimento a nível de complexidade das Infraestruturas de \textit{IT} assim como o aumento
das \textit{cyber threats} exigem cada vez mais o uso de práticas robustas de gestão de riscos para garantir
a segurança e resiliência dos sistemas num contexto empresarial. Com a implementação da \textit{NIS2
Directive} na União Europeia, tornou-se obrigatório a adoção de mecanismos de avaliação de risco
que assegurem conformidade regulatória e proteção eficiente dos seus ativos críticos.
Este trabalho tem como motivação explorar e comparar as principais ferramentas de avaliação de
risco disponíveis, com ênfase na sua eficiência operacional e na capacidade de cumprir os
requisitos impostos pela \textit{NIS2}. Através de uma análise comparativa, pretende-se identificar quais
as soluções que melhor se adaptam às necessidades de segurança das infraestruturas de hoje em
dia, nomeadamente \textit{Data Centers} e \textit{Enterprise Networks}. Este estudo visa para uma compreensão
aprofundada das melhores práticas de avaliação de risco que equilibram conformidade regulatória
e eficiência, permitindo às organizações adotar estratégias mais resilientes e alinhadas com os
requisitos de cibersegurança vigentes relativamente à infraestrutura.


\keywords{
  Infraestruturas de IT \and
  Cyber threats \and
  Gestão de Riscos \and
  Segurança \and
  Resiliência \and
  Contexto Empresarial \and  
  NIS2 Directive \and 
  União Europeia \and 
  Conformidade regulatória \and 
  Avaliação de risco \and 
  Ativos críticos \and 
  Efficiência operacional \and 
  Análise comparativa \and 
  Necessidades de segurança \and 
  Data Centers \and 
  Enterprise Networks \and 
  Cibersegurança \and 
  estratégias resilientes
}
